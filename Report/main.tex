\documentclass[a4paper, 12pt]{extarticle}


% Includes 
\usepackage[utf8]{inputenc} % UTF-8 ebcode 
\usepackage[english, russian]{babel}
\usepackage{geometry} % adjust paje layout 
\usepackage{graphicx} 
\usepackage{hyperref} 
\usepackage{amsmath} % math formulas 
\usepackage{setspace} % for set line spacing 
\usepackage{indentfirst} % indent on a first line after the paragraph 
\usepackage{pgfplots} % for plots 
\usepackage{listings} % for code listiongs 


% debug
% \usepackage{showframe} % frame borders for demontration 


% Settings for links 
\hypersetup{
    colorlinks,
    citecolor=black,
    filecolor=black,
    linkcolor=black,
    urlcolor=black
}


% Layouts 
\geometry{
	left=17mm,
	top=15mm,
	right=17mm,
	bottom=15mm,
	marginparsep=0mm,
	marginparwidth=0mm,
	headheight=10mm,
	headsep=7mm, 
	nofoot
}

\linespread{1.5} % line spacing
\setlength{\parskip}{\baselineskip}  % Add space between paragraphs


% overfull hbox settings
\tolerance 1400 % default 200, max 10000
\hbadness 1400 % default 1000, max 10000
\emergencystretch 0pt  % default 0pt, how much the lines can stretch for the sake of good line breaks
\hfuzz 0.4pt % ignore overfull box less than 
\widowpenalty=10000 % no lines at the start of the page
\vfuzz \hfuzz % don't care about underfull vbox if overfull is acceptable
\raggedbottom % if the page is not filled, align the content to the bottom

% Redefinition of table of contents command to get centered heading
\makeatletter
\renewcommand\tableofcontents{ 
  \begin{singlespace}
    \null\hfill\textbf{\Large\contentsname}\hfill\null\par
    \@mkboth{\MakeUppercase\contentsname}{\MakeUppercase\contentsname}%
    \@starttoc{toc}%
  \end{singlespace}
}
\makeatother

% Redefinition of section and subsection numbering style
\def\thesection{\arabic{section}.}
\def\thesubsection{\arabic{section}.\arabic{subsection}.}
\def\thesubsubsection{\arabic{section}.\arabic{subsection}.\arabic{subsubsection}.}


% commands for unnumbered sections
\newcommand{\usection}[1]{\section*{#1} \addcontentsline{toc}{section}{\protect\numberline{}#1}}
\newcommand{\usubsection}[1]{\subsection*{#1} \addcontentsline{toc}{subsection}{\protect\numberline{}#1}}
\newcommand{\usubsubsection}[1]{\subsubsection*{#1} \addcontentsline{toc}{subsubsection}{\protect\numberline{}#1}}

% Listings settings
\definecolor{codegreen}{rgb}{0, 0.6, 0}
\definecolor{codegray}{rgb}{0.5, 0.5, 0.5}
\definecolor{codepurple}{rgb}{0.58, 0, 0.82}
\definecolor{backcolour}{rgb}{0.98, 0.98, 0.98}

\lstdefinestyle{python}{
  language=Python,
  backgroundcolor=\color{backcolour},   
  commentstyle=\color{codegreen},
  keywordstyle=\color{blue},
  numberstyle=\tiny\color{codegray},
  stringstyle=\color{codepurple},
  basicstyle=\ttfamily\small,
  breakatwhitespace=true,         
  breaklines=true,                 
  captionpos=b, % t/b                  
  keepspaces=true,                 
  numbers=none, % none/left/rigth                    
  numbersep=5pt,                  
  showspaces=false,                
  showstringspaces=false,
  showtabs=false,                  
  tabsize=2,
  frame=shadowbox, % none/leftline/topline/bottomline/lines/single/shadowbox
  rulecolor=\color{gray}, % frame color 
}

\lstset{style=python}


%%% Content %%% 


% For title page 
\def\name{Отчет по лабораторной работе №1} 
\def\subname{Гистограммы, профили и проекции}
\def\madeby{Александр Иванов, R3238\\Никита Братушка, R3238\\Ани Аракелян, R3242}
\def\teacher{Шаветов С. В.}


\begin{document}

% include title page 
% Title page 
\begin{titlepage}

\thispagestyle{empty}

\title{
    \includegraphics[width=4cm]{media/logo.png} \\
    \vspace{1em}
    НИУ ИТМО 
    \vspace{4em}

    \begin{center}
        \large \textsc{\textbf{\name}}
        \\ \vspace{1em}
        ``\subname''  % Title 
    \end{center}

    \vspace*{\fill}

    \begin{flushright}
        {\normalsize 
            Выполнили: \\ \textbf{\madeby} \\
            \vspace{1em}
            Преподаватель: \\ \textbf{\teacher} \\
        }
    \end{flushright}	

    \vspace{2em}

    \begin{center}
        \small{Санкт-Петербург, \the\year}
    \end{center}
}

\author{}
\date{}
\maketitle
\thispagestyle{empty}
\end{titlepage}

\tableofcontents
\newpage

\section{Используемые функции}

\subsection{Функция для вычисления гистограммы изображения}

Для вычисления гистограммы изображений была написана слеюующая функция:

\begin{lstlisting}

def calc_hist_normalised(img):
  histSize = 256
  histRange = (0, 256)
  # calculate the histograms
  b_hist = cv2.calcHist([img], [0], None, [histSize], histRange) / (img.shape[0] * img.shape[1])
  g_hist = cv2.calcHist([img], [1], None, [histSize], histRange) / (img.shape[0] * img.shape[1])
  r_hist = cv2.calcHist([img], [2], None, [histSize], histRange) / (img.shape[0] * img.shape[1])

  return b_hist, g_hist, r_hist
\end{lstlisting}

В качестве аргумента она принимает изображение, для которого необходимо вычислить гистограмму. Возвращает функция нормализованные гистограммы для каждого канала изображения.
Под нормализованной гистограммой понимается гистограмма, в которой каждое значение поделено на общее количество пикселей в изображении.

\subsection{Функция для отображения изображения и его гистограмм}
Для отображения изображения, его гистограмм и результата преобразования использовалась следующая функция: 

\begin{lstlisting}[language=Python]
# add histogram and image to the same plot
def show_image_with_hist(img, title="Image"):
  img_rgb = cv2.cvtColor(img, cv2.COLOR_BGR2RGB) # normal colors to display
  b_hist, g_hist, r_hist = calc_hist_normalised(img) # calculate the histograms
  cumulative_hist_b,cumulative_hist_g, cumulative_hist_r = np.cumsum(b_hist), np.cumsum(g_hist), np.cumsum(r_hist) # calculate the cumulative histograms

  gs = plt.GridSpec(2, 4, width_ratios=[3, 1, 1, 1])

  plt.figure(figsize=(13, 5))

  plt.suptitle(title, fontsize=16)
  ax0 = plt.subplot(gs[:, 0]) # for image 
  ax0.set_title('Image')
  ax1 = plt.subplot(gs[0, 1:4]) # for rgb hist
  ax1.set_title('RGB Histogram')

  ax2 = plt.subplot(gs[1, 1:3])
  ax2.set_title('Cumulative RGB Histogram')
  ax3 = plt.subplot(gs[1, 3])
  ax3.set_title('Average Histogram')

  # set the x axis limits for the histogram subplots and grid
  for ax in [ax1, ax2, ax3]:
      ax.set_xlim([0, 256])
      ax.grid(True)

  # display the image
  ax0.imshow(img_rgb)
  ax0.axis('off')

  # all 3 histograms
  ax1.plot(b_hist, color='b')
  ax1.plot(g_hist, color='g')
  ax1.plot(r_hist, color='r')

  # cumulative histograms
  ax2.plot(cumulative_hist_b, color='b')
  ax2.plot(cumulative_hist_g, color='g')
  ax2.plot(cumulative_hist_r, color='r')

  # add 3 colors hist
  avg_hits = (b_hist + g_hist + r_hist) / 3
  ax3.hist(np.arange(0, 256), bins=256, weights=avg_hits, color='black')

  # adjust the layout
  plt.subplots_adjust(wspace=0.3, hspace=0.3)
  plt.subplots_adjust(left=0.05, right=0.95)  
\end{lstlisting}

Во всех следующих функциях для преобразования изображений оно сначала раскладывается на каналы, затем преобразование применяется к каждому каналу, после чего каналы собираются обратно в изображение. 
В некоторых случаях это позволяет примернить различные значения преобразования к различным каналам изображения, что позволяет получить более интересные результаты.

\section{Преобразования изображений}

\subsection{Исходное изображение}

\begin{figure}[h]
    \centering
    \includegraphics[width=\textwidth]{../results/Source image.png}
    \caption{Исходное изображение}
    \label{fig:source}
\end{figure}

На рисунке \ref{fig:source} представлено исходное изображение, для которого необходимо вычислить гистограмму.

Кроме того, в правой части рисунка \ref{fig:source} представлены гистограммы для каждого канала изображения, коммулятивная гистограмма и средняя гистограмма для всех каналов.

\subsection{Линейное выравнивание гистограммы}

Суть линейного выравнивания гистограммы заклбчается в исопльзовании коммулятивнной гистограммы для вычисления нового значения пикселя.

Исходный код функции для линейного выравнивания гистограммы:

\begin{lstlisting}[language=Python]
def linear_leveling_transformation(img):
  # split the image into its 3 channels
  b, g, r = cv2.split(img)

  # calculate the histograms
  hist = calc_hist_normalised(img)

  # calculate the cumulative histograms
  cumulative_histogram_b = np.cumsum(hist[0]) 
  cumulative_histogram_g = np.cumsum(hist[1])
  cumulative_histogram_r = np.cumsum(hist[2])

  # apply the transformation
  b = np.clip(255 * cumulative_histogram_b[b], 0, 255)
  g = np.clip(255 * cumulative_histogram_g[g], 0, 255)
  r = np.clip(255 * cumulative_histogram_r[r], 0, 255)

  # merge the channels back
  return cv2.merge([b, g, r]).astype(np.uint8)
\end{lstlisting}

\begin{figure}[h]
    \centering
    \includegraphics[width=\textwidth]{../results/Linear leveling transformation.png}
    \caption{Результат линейного выравнивания гистограммы}
    \label{fig:linear}
\end{figure}

На рисунке \ref{fig:linear} представлен результат применения линейного выравнивания гистограммы к исходному изображению. Как видно, коммулятивная гистограмма стала представлять из себя линейную функцию, что и является результатом применения линейного выравнивания гистограммы. 

Цвета изображения при этом стали более холодынми, сама гистограмма стала прерывистой, что связано с тем, что на преобразованном изображении не могут появиться некоторые значения пикселей. 

\subsection{Арифетические операции над изображениями}

Для увеличения детализации некоторых областей изображения, можно использовать арифметические операции над изображениями. Например, для увеличения детализации теней можно использовать смещение цветов изображения в сторону светлых тонов.

Исходный код функции для арифметических операций над изображениями:

\begin{lstlisting}[language=Python]
def aritmetic_transformation(img, b_delta, g_delta, r_delta):
  # split the image into its 3 channels
  b, g, r = cv2.split(img)

  # add the delta to each channel and divide by 255
  b = (b + b_delta) 
  g = (g + g_delta) 
  r = (r + r_delta) 

  # merge the channels back
  return cv2.merge([b, g, r])
\end{lstlisting}


Так как мы используем цветное изображение, то для каждого канала можно использовать свой коэффициент смещения.

\begin{figure}[h]
    \centering
    \includegraphics[width=\textwidth]{../results/Aritmetic transformation.png}
    \caption{Результат арифметических операций над изображением}
    \label{fig:aritmetic}
\end{figure}

На рисунке \ref{fig:aritmetic} представлен результат применения арифметических операций над изображением. К красному, зеленому и синему каналам изображения были добавлены значения 30, 20 и 10 соответственно.

Как видно, изображение стало более светлым, что связано с тем, что мы добавили ко всем каналам изображения некоторое значение.
Так же видны зелено-синие пятна на изображении, что связано с тем, что мы добавили ко всем каналам значения, что привело к изменению цветового баланса изображения.

На гистограмме так же видно, что на изображении стало больше светлых пикселей. 

\subsection{Растяжение динамического диапазона}

Данное преобразование выполняется согласно следующему закону: 

\begin{equation}
  I_{new} = \left( \frac{I - I_{min}}{I_{max} - I_{min}} \right)^\alpha
\end{equation}

где $I$ -- исходное изображение, $I_{new}$ -- преобразованное изображение, $I_{min}$ и $I_{max}$ -- минимальное и максимальное значение пикселя на изображении, $\alpha$ -- коэффициент нелинейности.

Исходный код функции для растяжения динамического диапазона:

\begin{lstlisting}[language=Python]
def dynamic_range_expansion_transformation(img, aplha):
  # find maximum and minimum values for each channel

  # if the image is of type uint8, convert it to float64
  if img.dtype == 'uint8':
      img_converted = img.astype(np.float64) / 255

  b, g, r = cv2.split(img_converted)
  b_min, b_max = b.min(), b.max()
  g_min, g_max = g.min(), g.max()
  r_min, r_max = r.min(), r.max()

  # apply the transformation
  b = np.clip(((b - b_min) / (b_max - b_min)) ** aplha, 0, 1)
  g = np.clip(((g - g_min) / (g_max - g_min)) ** aplha, 0, 1)
  r = np.clip(((r - r_min) / (r_max - r_min)) ** aplha, 0, 1)

  img_transformed = cv2.merge([b, g, r]) # merge the channels back

  # convert the image back to uint8 if it was initially of that type
  if img.dtype == 'uint8':
      img_transformed = (255 * img_transformed).clip(0, 255).astype(np.uint8)

  return img_transformed
\end{lstlisting}

\begin{figure}[h]
    \centering
    \includegraphics[width=\textwidth]{../results/Dynamic range expansion.png}
    \caption{Результат растяжения динамического диапазона}
    \label{fig:dynamic}
\end{figure}

На рисунке \ref{fig:dynamic} представлен результат применения растяжения динамического диапазона к исходному изображению. Как видно, изображение стало более контрастным. 

\subsection{Равномерное преобразование}

Равномерное преобразование выполняется согласно следующему закону:

\begin{equation}
  I_{new} = \left( I_{max} - I_{min} \right) * P(I) + I_{min}
\end{equation}

где $I$ -- исходное изображение, $I_{new}$ -- преобразованное изображение, $I_{min}$ и $I_{max}$ -- минимальное и максимальное значение пикселя на изображении, $P(I)$ -- функция распределения вероятностей исходонго изображения, которая аппроксимируется коммулятивнной гистограммой: 

\begin{equation}
  P(I) \approx \sum\limits_{m=0}^{i} Hist(m)
\end{equation}

Исходный код функции для равномерного преобразования:

\begin{lstlisting}[language=Python]
def uniform_transformation(img):
  b, g, r = cv2.split(img)
  # calculate the histograms
  hist = calc_hist_normalised(img)

  b_min, b_max = b.min(), b.max()
  g_min, g_max = g.min(), g.max()
  r_min, r_max = r.min(), r.max()

  cumulative_histogram_b = np.cumsum(hist[0]) 
  cumulative_histogram_g = np.cumsum(hist[1])
  cumulative_histogram_r = np.cumsum(hist[2])

  b = np.clip((b_max - b_min) * cumulative_histogram_b[b] + b_min, 0, 255)
  g = np.clip((g_max - g_min) * cumulative_histogram_g[g] + g_min, 0, 255)
  r = np.clip((r_max - r_min) * cumulative_histogram_r[r] + r_min, 0, 255)

  return cv2.merge([b, g, r]).astype(np.uint8)
\end{lstlisting}

\begin{figure}[h]
    \centering
    \includegraphics[width=\textwidth]{../results/Uniform transformation.png}
    \caption{Результат равномерного преобразования}
    \label{fig:uniform}
\end{figure}

На рисунке \ref{fig:uniform} представлен результат применения равномерного преобразования к исходному изображению. Как видно, изображение стало более контрастным, при этом коммулятивнная гистограмма стала представлять из себя линейную функцию, что схоже с результатом применения линейного выравнивания гистограммы.


\subsection{Экспоненциальное преобразование}

Экспоненциальное преобразование выполняется согласно следующему закону:

\begin{equation}
  I_{new} = I_min - \frac{1}{\alpha} \cdot \ln(1 - P(I))
\end{equation}

где $I$ -- исходное изображение, $I_{new}$ -- преобразованное изображение, $I_{min}$ -- минимальное значение пикселя на изображении, $P(I)$ -- функция распределения вероятностей исходонго изображения, $\alpha$ -- постоянная, характеризующая крутизну преобразования. 

Исходный код функции для экспоненциального преобразования:

\begin{lstlisting}[language=Python]
def exponentional_transformation(img, alpha):
  b, g, r = cv2.split(img)
  # calculate the histograms
  hist = calc_hist_normalised(img)

  b_min, b_max = b.min(), b.max()
  g_min, g_max = g.min(), g.max()
  r_min, r_max = r.min(), r.max()

  cumulative_histogram_b = np.cumsum(hist[0]) 
  cumulative_histogram_g = np.cumsum(hist[1])
  cumulative_histogram_r = np.cumsum(hist[2])

  b = b_min - 255/alpha * np.log(1 - cumulative_histogram_b[b]) 
  g = g_min - 255/alpha * np.log(1 - cumulative_histogram_g[g]) 
  r = r_min - 255/alpha * np.log(1 - cumulative_histogram_r[r]) 

return (cv2.merge([b, g, r])).astype(np.uint8).clip(0, 255)
\end{lstlisting}

\begin{figure}[h]
    \centering
    \includegraphics[width=\textwidth]{../results/Exponentional transformation.png}
    \caption{Результат экспоненциального преобразования}
    \label{fig:exponentional}
\end{figure}

На рисунке \ref{fig:exponentional} видим, что график коммулятивнной гистограммы действительно стал представлять из себя экспоненциальную функцию. 

При этом изображение стало значительно темнее, что так же видно на гистограммах -- темные пиксели стали преобладать на изображении.


\subsection{Преобразование по закону Рэлея}  

Преобразование по закону Рэлея выполняется согласно следующему закону:

\begin{equation}
  I_{new} = I_{min} + \left(2\alpha^2 \cdot \ln\left(\frac{1}{1 - P(I)}\right)\right)
\end{equation}

где $\alpha$ -- постоянная, характеризующая гистограмму распределения интенсивностей элементов результирующего изображения. 

Исходный код функции для преобразования по закону Рэлея:

\begin{lstlisting}[language=Python]
def rayleigh_law_transformation(img, alpha):
  b, g, r = cv2.split(img)
  # calculate the histograms
  hist = calc_hist_normalised(img)

  b_min, b_max = b.min(), b.max()
  g_min, g_max = g.min(), g.max()
  r_min, r_max = r.min(), r.max()

  cumulative_histogram_b = np.cumsum(hist[0]) 
  cumulative_histogram_g = np.cumsum(hist[1])
  cumulative_histogram_r = np.cumsum(hist[2])

  b = np.clip(b_min + (2*alpha**2 * np.log(1 / (1 - cumulative_histogram_b[b]))) ** 0.5 * 255, 0, 255)
  g = np.clip(g_min + (2*alpha**2 * np.log(1 / (1 - cumulative_histogram_g[g]))) ** 0.5 * 255, 0, 255)
  r = np.clip(r_min + (2*alpha**2 * np.log(1 / (1 - cumulative_histogram_r[r]))) ** 0.5 * 255, 0, 255)
  
  return cv2.merge([b, g, r]).astype(np.uint8)
\end{lstlisting}

\begin{figure}[h]
    \centering
    \includegraphics[width=\textwidth]{../results/Rayleigh law transformation.png}
    \caption{Результат преобразования по закону Рэлея}
    \label{fig:rayleigh}
\end{figure}

На рисунке \ref{fig:rayleigh} видим, что график коммулятивнной гистограммы действительно стал представлять из себя функцию Рэлея. Так же наблюдаем сильное преобладание светлых пикселей на изображении. 

\subsection{Преобразование по закону степени 2/3}

Преобразование по закону степени 2/3 выполняется согласно следующему закону:

\begin{equation}
  I_{new} = P(I)^{\frac{3}{2}}
\end{equation}

Исходный код функции для преобразования по закону степени 2/3:

\begin{lstlisting}[language=Python]
def two_thirds_rule_transformation(img):
  b, g, r = cv2.split(img)
  # calculate the histograms
  hist = calc_hist_normalised(img)

  b_min, b_max = b.min(), b.max()
  g_min, g_max = g.min(), g.max()
  r_min, r_max = r.min(), r.max()

  cumulative_histogram_b = np.cumsum(hist[0]) 
  cumulative_histogram_g = np.cumsum(hist[1])
  cumulative_histogram_r = np.cumsum(hist[2])

  b = np.clip((cumulative_histogram_b[b] ** (2/3) * 255), 0, 255)
  g = np.clip((cumulative_histogram_g[g] ** (2/3) * 255), 0, 255)
  r = np.clip((cumulative_histogram_r[r] ** (2/3) * 255), 0, 255)
  
  return cv2.merge([b, g, r]).astype(np.uint8)
\end{lstlisting}

\begin{figure}[h]
    \centering
    \includegraphics[width=\textwidth]{../results/Two thirds rule transformation.png}
    \caption{Результат преобразования по закону степени 2/3}
    \label{fig:two_thirds}
\end{figure}

На рисунке \ref{fig:two_thirds} видим, что график коммулятивнной гистограммы действительно стал представлять из себя функцию степени 2/3. 

изображение стало намного более светлым, контрастноть упала. 

\newpage
\subsection{Гиперболическое преобразование}  

Гиперболическое преобразование выполняется согласно следующему закону:

\begin{equation}
  I_{new} = \alpha^{P(I)}
\end{equation}

где $\alpha$ -- постоянная, относительно которой осуществляется преобразование.

Исходный код функции для гиперболического преобразования:

\begin{lstlisting}[language=Python]
def hyperbolic_transformation(img, alpha):
  b, g, r = cv2.split(img)
  # calculate the histograms
  hist = calc_hist_normalised(img)

  b_min, b_max = b.min(), b.max()
  g_min, g_max = g.min(), g.max()
  r_min, r_max = r.min(), r.max()

  cumulative_histogram_b = np.cumsum(hist[0]) 
  cumulative_histogram_g = np.cumsum(hist[1])
  cumulative_histogram_r = np.cumsum(hist[2])

  b = np.clip(alpha ** cumulative_histogram_b[b] * 255, 0, 255)
  g = np.clip(alpha ** cumulative_histogram_g[g] * 255, 0, 255)
  r = np.clip(alpha ** cumulative_histogram_r[r] * 255, 0, 255)
  
  return cv2.merge([b, g, r]).astype(np.uint8)
\end{lstlisting}

\begin{figure}[h]
    \centering
    \includegraphics[width=\textwidth]{../results/Hyperbolic transformation.png}
    \caption{Результат гиперболического преобразования}
    \label{fig:hyperbolic}
\end{figure}

На рисунке \ref{fig:hyperbolic} видим, что график коммулятивнной гистограммы действительно стал представлять из себя гиперболу. 

Изображение можно описать как негативное от исходонго, заметно явное преобладание темных пикселей на изображении.

\subsection{Таблица поиска}

Таблица поиска (lookup table) -- это таблица, в которой для каждого значения пикселя изображения задается новое значение пикселя.

Исходный код функции для преобразования по таблице поиска:

\begin{lstlisting}[language=Python]
def LUT_transformation(img, LUT):
  return cv2.LUT(img, LUT).astype(np.uint8)
\end{lstlisting}

Для создания самой таблицы поиска использовался следующий код:  

\begin{lstlisting}[language=Python]
# generate LUT 
lut = np.arange(256, dtype = np.uint8)
lut = np.clip(np.power(lut, 0.9) + 20, 0, 255)
\end{lstlisting}

\begin{figure}[h]
    \centering
    \includegraphics[width=\textwidth]{../results/LUT transformation.png}
    \caption{Результат преобразования по таблице поиска}
    \label{fig:lut}
\end{figure}

На рисунке \ref{fig:lut} видим, что график коммулятивнной гистограммы стал представлять из себя функцию степени 0.9, смешение на 20, что и было задано в таблице поиска.  

\newpage
\section{Прифили изображения}

Прифилем изображения вдоль некоторой линии называется функция интенсивности изображения, распределенного вдоль этой линии (\textit{прорезки}).

Рассмотрим профили изображения вдоль горизонтальной и вертикальной линий: 

\begin{equation}
  Profile~i(x) = I(x, i)
\end{equation}
  
\begin{equation}
  Profile~i(y) = I(i, y)
\end{equation}

где $I(x, i)$ -- интенсивность пикселя изображения в точке $(x, i)$, $I(i, y)$ -- интенсивность пикселя изображения в точке $(i, y)$.

\subsection{Профиль изображения вдоль горизонтальной линии}

Рассмотрим проифиль изображения со штрих-кодом вдоль горизонтальной линии.

Исходный код функции для вычисления профиля изображения вдоль горизонтальной линии:

\begin{lstlisting}[language=Python]
def show_image_profile(img, level):
  img_rgb = cv2.cvtColor(img, cv2.COLOR_BGR2RGB) # normal colors to display
  plt.figure(figsize=(13, 5))
  plt.subplot(1, 2, 1)
  plt.title('Image')
  plt.imshow(img_rgb)
  plt.axis('off') # disable the axis

  plt.subplot(1, 2, 2)
  profile = img[level, :]
  plt.plot(profile)
  plt.title('Profile')

  plt.subplots_adjust(wspace=0.3, hspace=0.3)
  plt.subplots_adjust(left=0.05, right=0.95)

show_image_profile(img, img.shape[0] // 2)
\end{lstlisting}

\begin{figure}[h]
    \centering
    \includegraphics[width=\textwidth]{../results/Profile.png}
    \caption{Профиль изображения вдоль горизонтальной линии}
    \label{fig:profile}
\end{figure}

Видим, что на проекции хорошо различимы  \textit{штрихи} штрих-кода. Данную проекцию можно исопльзовать для распознавания штрих-кода. 

\section{Проекция изображения}

Проекцией изображения на некоторую ось называется сумма интенсивностей пикселей изображения в направлении, перпендикулярном этой оси.

\subsection{Проекция изображения на оси}

Рассмотрим проекцию изображения на горизонтальную и вертикальную оси:

\begin{equation}
  Projection~X(y) = \sum\limits_{i=0}^{dim~Y - 1} I(y, i)
\end{equation}

\begin{equation}
  Projection~Y(x) = \sum\limits_{i=0}^{dim~X - 1} I(i, x)
\end{equation}

Исходный код функции для вычисления проекции изображения на оси:

\begin{lstlisting}[language=Python]
def show_image_projection(img, title="Projection"):
  b, g, r = cv2.split(img)
  # calculate 0x projection
  projection_x = (np.sum(b, axis=0) + np.sum(g, axis=0) + np.sum(r, axis=0)) / img.shape[0] / 3

  # calculate 0y projection
  projection_y = (np.sum(b, axis=1) + np.sum(g, axis=1) + np.sum(r, axis=1)) / img.shape[1] / 3

  plt.figure(figsize=(8, 8))
  plt.subplot(2, 2, 1)
  plt.grid(True)
  plt.title('Image')
  plt.imshow(cv2.cvtColor(img, cv2.COLOR_BGR2RGB))

  plt.subplot(2, 2, 3)
  plt.ylim([255, 0])
  plt.xlim([0, img.shape[1]])
  plt.grid(True)
  plt.plot(range(img.shape[1]), projection_x, color='g')
  plt.title('Projection X')

  plt.subplot(2, 2, 2)
  plt.xlim([0, 255])
  plt.ylim([img.shape[1], 0])
  plt.grid(True)
  plt.plot(projection_y, range(img.shape[0]), color='g')
  plt.title('Projection Y')

  plt.subplots_adjust(wspace=0.3, hspace=0.3)
  plt.subplots_adjust(left=0.05, right=0.95)

  # save image
  plt.savefig(f"results/{title}.png")

show_image_projection(img1, "Projection 1")
show_image_projection(img2, "Projection 2")
\end{lstlisting}

\begin{figure}[h]
    \centering
    \includegraphics[width=\textwidth]{../results/Projection 1.png}
    \caption{Проекция изображения 1}
    \label{fig:projection1}
\end{figure}

\begin{figure}[h]
    \centering
    \includegraphics[width=\textwidth]{../results/Projection 2.png}
    \caption{Проекция изображения 2}
    \label{fig:projection2}
\end{figure}

На рисунках \ref{fig:projection1} и \ref{fig:projection2} представлены проекции изображений на горизонтальную и вертикальную оси. 
По этим проекциям можно определить расположение объектов на изображении, а так же их размеры.

\newpage\clearpage
\section{Вывод}

Exercitation non cupidatat exercitation velit nostrud dolor. Ipsum dolor elit ex et minim enim. Occaecat occaecat eu eu cillum pariatur velit non anim culpa. Esse officia magna duis aliqua dolore. In anim commodo consectetur ullamco fugiat Lorem magna nisi.

\section{Ответы на вопросы}

\newcounter{question}
\setcounter{question}{0}

\newcommand{\question}[1]{\item[Q\refstepcounter{question}\thequestion.] #1}
\newcommand{\answer}[1]{\item[A\thequestion.] #1}

\begin{itemize}
  \question{Что такое контрастность изозбражения и как ее можно изверить?}
  \answer{Контранстность -- разность между цветом самого темного и самого светлого пикселя на изображении}

  \question{Чем эффективно использование профилей и проекций изображения?}
  \answer{С помощью профилей и проекций можно определить расположение объектов на изображении, а так же их размеры}

  \question{Каким образом можной найти объект на равномерном фоне?}
  \answer{Можно построить проекцию изображения на оси и найти пики на графиках проекций. Это будет середине объектов на изображении.}
\end{itemize}


\end{document}
